% Options for packages loaded elsewhere
\PassOptionsToPackage{unicode}{hyperref}
\PassOptionsToPackage{hyphens}{url}
%
\documentclass[
]{article}
\usepackage{amsmath,amssymb}
\usepackage{iftex}
\ifPDFTeX
  \usepackage[T1]{fontenc}
  \usepackage[utf8]{inputenc}
  \usepackage{textcomp} % provide euro and other symbols
\else % if luatex or xetex
  \usepackage{unicode-math} % this also loads fontspec
  \defaultfontfeatures{Scale=MatchLowercase}
  \defaultfontfeatures[\rmfamily]{Ligatures=TeX,Scale=1}
\fi
\usepackage{lmodern}
\ifPDFTeX\else
  % xetex/luatex font selection
\fi
% Use upquote if available, for straight quotes in verbatim environments
\IfFileExists{upquote.sty}{\usepackage{upquote}}{}
\IfFileExists{microtype.sty}{% use microtype if available
  \usepackage[]{microtype}
  \UseMicrotypeSet[protrusion]{basicmath} % disable protrusion for tt fonts
}{}
\makeatletter
\@ifundefined{KOMAClassName}{% if non-KOMA class
  \IfFileExists{parskip.sty}{%
    \usepackage{parskip}
  }{% else
    \setlength{\parindent}{0pt}
    \setlength{\parskip}{6pt plus 2pt minus 1pt}}
}{% if KOMA class
  \KOMAoptions{parskip=half}}
\makeatother
\usepackage{xcolor}
\usepackage[margin=1in]{geometry}
\usepackage{color}
\usepackage{fancyvrb}
\newcommand{\VerbBar}{|}
\newcommand{\VERB}{\Verb[commandchars=\\\{\}]}
\DefineVerbatimEnvironment{Highlighting}{Verbatim}{commandchars=\\\{\}}
% Add ',fontsize=\small' for more characters per line
\usepackage{framed}
\definecolor{shadecolor}{RGB}{248,248,248}
\newenvironment{Shaded}{\begin{snugshade}}{\end{snugshade}}
\newcommand{\AlertTok}[1]{\textcolor[rgb]{0.94,0.16,0.16}{#1}}
\newcommand{\AnnotationTok}[1]{\textcolor[rgb]{0.56,0.35,0.01}{\textbf{\textit{#1}}}}
\newcommand{\AttributeTok}[1]{\textcolor[rgb]{0.13,0.29,0.53}{#1}}
\newcommand{\BaseNTok}[1]{\textcolor[rgb]{0.00,0.00,0.81}{#1}}
\newcommand{\BuiltInTok}[1]{#1}
\newcommand{\CharTok}[1]{\textcolor[rgb]{0.31,0.60,0.02}{#1}}
\newcommand{\CommentTok}[1]{\textcolor[rgb]{0.56,0.35,0.01}{\textit{#1}}}
\newcommand{\CommentVarTok}[1]{\textcolor[rgb]{0.56,0.35,0.01}{\textbf{\textit{#1}}}}
\newcommand{\ConstantTok}[1]{\textcolor[rgb]{0.56,0.35,0.01}{#1}}
\newcommand{\ControlFlowTok}[1]{\textcolor[rgb]{0.13,0.29,0.53}{\textbf{#1}}}
\newcommand{\DataTypeTok}[1]{\textcolor[rgb]{0.13,0.29,0.53}{#1}}
\newcommand{\DecValTok}[1]{\textcolor[rgb]{0.00,0.00,0.81}{#1}}
\newcommand{\DocumentationTok}[1]{\textcolor[rgb]{0.56,0.35,0.01}{\textbf{\textit{#1}}}}
\newcommand{\ErrorTok}[1]{\textcolor[rgb]{0.64,0.00,0.00}{\textbf{#1}}}
\newcommand{\ExtensionTok}[1]{#1}
\newcommand{\FloatTok}[1]{\textcolor[rgb]{0.00,0.00,0.81}{#1}}
\newcommand{\FunctionTok}[1]{\textcolor[rgb]{0.13,0.29,0.53}{\textbf{#1}}}
\newcommand{\ImportTok}[1]{#1}
\newcommand{\InformationTok}[1]{\textcolor[rgb]{0.56,0.35,0.01}{\textbf{\textit{#1}}}}
\newcommand{\KeywordTok}[1]{\textcolor[rgb]{0.13,0.29,0.53}{\textbf{#1}}}
\newcommand{\NormalTok}[1]{#1}
\newcommand{\OperatorTok}[1]{\textcolor[rgb]{0.81,0.36,0.00}{\textbf{#1}}}
\newcommand{\OtherTok}[1]{\textcolor[rgb]{0.56,0.35,0.01}{#1}}
\newcommand{\PreprocessorTok}[1]{\textcolor[rgb]{0.56,0.35,0.01}{\textit{#1}}}
\newcommand{\RegionMarkerTok}[1]{#1}
\newcommand{\SpecialCharTok}[1]{\textcolor[rgb]{0.81,0.36,0.00}{\textbf{#1}}}
\newcommand{\SpecialStringTok}[1]{\textcolor[rgb]{0.31,0.60,0.02}{#1}}
\newcommand{\StringTok}[1]{\textcolor[rgb]{0.31,0.60,0.02}{#1}}
\newcommand{\VariableTok}[1]{\textcolor[rgb]{0.00,0.00,0.00}{#1}}
\newcommand{\VerbatimStringTok}[1]{\textcolor[rgb]{0.31,0.60,0.02}{#1}}
\newcommand{\WarningTok}[1]{\textcolor[rgb]{0.56,0.35,0.01}{\textbf{\textit{#1}}}}
\usepackage{graphicx}
\makeatletter
\def\maxwidth{\ifdim\Gin@nat@width>\linewidth\linewidth\else\Gin@nat@width\fi}
\def\maxheight{\ifdim\Gin@nat@height>\textheight\textheight\else\Gin@nat@height\fi}
\makeatother
% Scale images if necessary, so that they will not overflow the page
% margins by default, and it is still possible to overwrite the defaults
% using explicit options in \includegraphics[width, height, ...]{}
\setkeys{Gin}{width=\maxwidth,height=\maxheight,keepaspectratio}
% Set default figure placement to htbp
\makeatletter
\def\fps@figure{htbp}
\makeatother
\setlength{\emergencystretch}{3em} % prevent overfull lines
\providecommand{\tightlist}{%
  \setlength{\itemsep}{0pt}\setlength{\parskip}{0pt}}
\setcounter{secnumdepth}{-\maxdimen} % remove section numbering
\ifLuaTeX
  \usepackage{selnolig}  % disable illegal ligatures
\fi
\usepackage{bookmark}
\IfFileExists{xurl.sty}{\usepackage{xurl}}{} % add URL line breaks if available
\urlstyle{same}
\hypersetup{
  pdftitle={Module Project 1: Infographic Assignment},
  pdfauthor={ENTER YOUR NAME HERE},
  hidelinks,
  pdfcreator={LaTeX via pandoc}}

\title{Module Project 1: Infographic Assignment}
\usepackage{etoolbox}
\makeatletter
\providecommand{\subtitle}[1]{% add subtitle to \maketitle
  \apptocmd{\@title}{\par {\large #1 \par}}{}{}
}
\makeatother
\subtitle{INSH 5302 - Information Design \& Visual Analytics}
\author{ENTER YOUR NAME HERE}
\date{January 21, 2024}

\begin{document}
\maketitle

{
\setcounter{tocdepth}{2}
\tableofcontents
}
\begin{Shaded}
\begin{Highlighting}[]
\CommentTok{\# Required packages for our course. Do not delete.}
\FunctionTok{library}\NormalTok{(tidyverse)}
\FunctionTok{library}\NormalTok{(mosaic)}
\end{Highlighting}
\end{Shaded}

\section{Big-picture}\label{big-picture}

\subsection{Research Question}\label{research-question}

\begin{quote}
How have literacy rates among young women in Liberia evolved over time,
and what trends can be observed in gender parity in education during the
same period?
\end{quote}

\begin{center}\rule{0.5\linewidth}{0.5pt}\end{center}

\section{Data}\label{data}

\subsection{Description of data}\label{description-of-data}

\begin{quote}
The dataset focuses on gender equality metrics in Liberia, detailing
indicators like the proportion of firms with female top managers, female
ownership participation, and youth literacy rates (specifically for
females aged 15-24). It also includes the gender parity index for youth
literacy, highlighting disparities in literacy rates between genders.
This dataset is sourced from the Humanitarian Data Exchange (HDX), an
open platform for sharing data across various crises and organizations.
HDX aims to simplify the discovery and usage of humanitarian data for
analysis \url{https://data.humdata.org/faq}.
\end{quote}

\subsection{Load data into R}\label{load-data-into-r}

\begin{Shaded}
\begin{Highlighting}[]
\NormalTok{gender\_lbr }\OtherTok{\textless{}{-}} \FunctionTok{read.csv}\NormalTok{(}\StringTok{"gender\_lbr.csv"}\NormalTok{)}
\FunctionTok{head}\NormalTok{(gender\_lbr)}
\end{Highlighting}
\end{Shaded}

\begin{verbatim}
##    Country.Name  Country.ISO3       Year                                            Indicator.Name       Indicator.Code
## 1 #country+name #country+code #date+year                                           #indicator+name      #indicator+code
## 2       Liberia           LBR       2017                Firms with female top manager (% of firms)       IC.FRM.FEMM.ZS
## 3       Liberia           LBR       2009                Firms with female top manager (% of firms)       IC.FRM.FEMM.ZS
## 4       Liberia           LBR       2017 Firms with female participation in ownership (% of firms)       IC.FRM.FEMO.ZS
## 5       Liberia           LBR       2009 Firms with female participation in ownership (% of firms)       IC.FRM.FEMO.ZS
## 6       Liberia           LBR       2019     Literacy rate, youth female (% of females ages 15-24) SE.ADT.1524.LT.FE.ZS
##                  Value
## 1 #indicator+value+num
## 2                 20.4
## 3                 29.9
## 4                 37.4
## 5                   53
## 6     71.8499984741211
\end{verbatim}

\begin{center}\rule{0.5\linewidth}{0.5pt}\end{center}

\section{Variables}\label{variables}

\begin{Shaded}
\begin{Highlighting}[]
\FunctionTok{names}\NormalTok{(gender\_lbr)}
\end{Highlighting}
\end{Shaded}

\begin{verbatim}
## [1] "Country.Name"   "Country.ISO3"   "Year"           "Indicator.Name" "Indicator.Code" "Value"
\end{verbatim}

The variables I used in my infographic design are:

\begin{enumerate}
\def\labelenumi{\arabic{enumi}.}
\tightlist
\item
  Year
\item
  Indicator.Name
\item
  Value
\end{enumerate}

\begin{center}\rule{0.5\linewidth}{0.5pt}\end{center}

\section{Data Analysis}\label{data-analysis}

\subsection{Summary Statistics}\label{summary-statistics}

\begin{Shaded}
\begin{Highlighting}[]
\CommentTok{\# Inspect the data}
\FunctionTok{summary}\NormalTok{(gender\_lbr)}
\end{Highlighting}
\end{Shaded}

\begin{verbatim}
##  Country.Name       Country.ISO3           Year           Indicator.Name     Indicator.Code        Value          
##  Length:5644        Length:5644        Length:5644        Length:5644        Length:5644        Length:5644       
##  Class :character   Class :character   Class :character   Class :character   Class :character   Class :character  
##  Mode  :character   Mode  :character   Mode  :character   Mode  :character   Mode  :character   Mode  :character
\end{verbatim}

\begin{Shaded}
\begin{Highlighting}[]
\CommentTok{\# Filter the data for literacy rates and gender parity index}
\NormalTok{literacy\_data }\OtherTok{\textless{}{-}} \FunctionTok{subset}\NormalTok{(gender\_lbr, Indicator.Name }\SpecialCharTok{==} \StringTok{"Literacy rate, youth female (\% of females ages 15{-}24)"}\NormalTok{)}
\NormalTok{gender\_parity\_data }\OtherTok{\textless{}{-}} \FunctionTok{subset}\NormalTok{(gender\_lbr, Indicator.Name }\SpecialCharTok{==} \StringTok{"Literacy rate, youth (ages 15{-}24), gender parity index (GPI)"}\NormalTok{)}

\CommentTok{\#convert to numeric}
\NormalTok{literacy\_data}\SpecialCharTok{$}\NormalTok{Year }\OtherTok{\textless{}{-}} \FunctionTok{as.numeric}\NormalTok{(literacy\_data}\SpecialCharTok{$}\NormalTok{Year)}
\NormalTok{literacy\_data}\SpecialCharTok{$}\NormalTok{Value }\OtherTok{\textless{}{-}} \FunctionTok{as.numeric}\NormalTok{(}\FunctionTok{gsub}\NormalTok{(}\StringTok{","}\NormalTok{, }\StringTok{""}\NormalTok{, literacy\_data}\SpecialCharTok{$}\NormalTok{Value)) }

\NormalTok{gender\_parity\_data}\SpecialCharTok{$}\NormalTok{Year }\OtherTok{\textless{}{-}} \FunctionTok{as.numeric}\NormalTok{(gender\_parity\_data}\SpecialCharTok{$}\NormalTok{Year)}
\NormalTok{gender\_parity\_data}\SpecialCharTok{$}\NormalTok{Value }\OtherTok{\textless{}{-}} \FunctionTok{as.numeric}\NormalTok{(}\FunctionTok{gsub}\NormalTok{(}\StringTok{","}\NormalTok{, }\StringTok{""}\NormalTok{, gender\_parity\_data}\SpecialCharTok{$}\NormalTok{Value)) }

\CommentTok{\#Checked for NA values to ensure data integrity.}
\FunctionTok{sum}\NormalTok{(}\FunctionTok{is.na}\NormalTok{(literacy\_data}\SpecialCharTok{$}\NormalTok{Year))}
\end{Highlighting}
\end{Shaded}

\begin{verbatim}
## [1] 0
\end{verbatim}

\begin{Shaded}
\begin{Highlighting}[]
\FunctionTok{sum}\NormalTok{(}\FunctionTok{is.na}\NormalTok{(literacy\_data}\SpecialCharTok{$}\NormalTok{Value))}
\end{Highlighting}
\end{Shaded}

\begin{verbatim}
## [1] 0
\end{verbatim}

\begin{Shaded}
\begin{Highlighting}[]
\FunctionTok{sum}\NormalTok{(}\FunctionTok{is.na}\NormalTok{(gender\_parity\_data}\SpecialCharTok{$}\NormalTok{Year))}
\end{Highlighting}
\end{Shaded}

\begin{verbatim}
## [1] 0
\end{verbatim}

\begin{Shaded}
\begin{Highlighting}[]
\FunctionTok{sum}\NormalTok{(}\FunctionTok{is.na}\NormalTok{(gender\_parity\_data}\SpecialCharTok{$}\NormalTok{Value))}
\end{Highlighting}
\end{Shaded}

\begin{verbatim}
## [1] 0
\end{verbatim}

Summary Statistics *Using the favstats() function

\begin{Shaded}
\begin{Highlighting}[]
\NormalTok{statistics1 }\OtherTok{\textless{}{-}} \FunctionTok{favstats}\NormalTok{(Value }\SpecialCharTok{\textasciitilde{}}\NormalTok{ Year, }\AttributeTok{data =}\NormalTok{ literacy\_data )}
\NormalTok{statistics2 }\OtherTok{\textless{}{-}} \FunctionTok{favstats}\NormalTok{(Value }\SpecialCharTok{\textasciitilde{}}\NormalTok{ Year, }\AttributeTok{data =}\NormalTok{ literacy\_data )}
\FunctionTok{head}\NormalTok{(statistics1)}
\end{Highlighting}
\end{Shaded}

\begin{verbatim}
##   Year      min       Q1   median       Q3      max     mean sd n missing
## 1 1984 33.72456 33.72456 33.72456 33.72456 33.72456 33.72456 NA 1       0
## 2 2007 37.17031 37.17031 37.17031 37.17031 37.17031 37.17031 NA 1       0
## 3 2013 63.20000 63.20000 63.20000 63.20000 63.20000 63.20000 NA 1       0
## 4 2017 45.63871 45.63871 45.63871 45.63871 45.63871 45.63871 NA 1       0
## 5 2019 71.85000 71.85000 71.85000 71.85000 71.85000 71.85000 NA 1       0
\end{verbatim}

\begin{Shaded}
\begin{Highlighting}[]
\FunctionTok{head}\NormalTok{(statistics2)}
\end{Highlighting}
\end{Shaded}

\begin{verbatim}
##   Year      min       Q1   median       Q3      max     mean sd n missing
## 1 1984 33.72456 33.72456 33.72456 33.72456 33.72456 33.72456 NA 1       0
## 2 2007 37.17031 37.17031 37.17031 37.17031 37.17031 37.17031 NA 1       0
## 3 2013 63.20000 63.20000 63.20000 63.20000 63.20000 63.20000 NA 1       0
## 4 2017 45.63871 45.63871 45.63871 45.63871 45.63871 45.63871 NA 1       0
## 5 2019 71.85000 71.85000 71.85000 71.85000 71.85000 71.85000 NA 1       0
\end{verbatim}

\begin{Shaded}
\begin{Highlighting}[]
\FunctionTok{ggplot}\NormalTok{(}\AttributeTok{data =}\NormalTok{ literacy\_data, }\FunctionTok{aes}\NormalTok{(}\AttributeTok{x =}\NormalTok{ Year, }\AttributeTok{y =}\NormalTok{ Value)) }\SpecialCharTok{+}
    \FunctionTok{geom\_line}\NormalTok{(}\AttributeTok{color =} \StringTok{"red"}\NormalTok{) }\SpecialCharTok{+} 
    \FunctionTok{theme\_minimal}\NormalTok{() }\SpecialCharTok{+}
    \FunctionTok{labs}\NormalTok{(}\AttributeTok{title =} \StringTok{"Trend of Female Literacy Rates in Liberia (Ages 15{-}24)"}\NormalTok{, }\AttributeTok{x =} \StringTok{"Year"}\NormalTok{, }\AttributeTok{y =} \StringTok{"Literacy Rate (\%)"}\NormalTok{)}
\end{Highlighting}
\end{Shaded}

\includegraphics{P1-final_submission-template--1-_files/figure-latex/unnamed-chunk-7-1.pdf}

\begin{Shaded}
\begin{Highlighting}[]
\FunctionTok{ggplot}\NormalTok{(}\AttributeTok{data =}\NormalTok{ gender\_parity\_data, }\FunctionTok{aes}\NormalTok{(}\AttributeTok{x =}\NormalTok{ Year, }\AttributeTok{y =}\NormalTok{ Value)) }\SpecialCharTok{+}
    \FunctionTok{geom\_line}\NormalTok{(}\AttributeTok{color =} \StringTok{"blue"}\NormalTok{) }\SpecialCharTok{+} 
    \FunctionTok{theme\_minimal}\NormalTok{() }\SpecialCharTok{+}
    \FunctionTok{labs}\NormalTok{(}\AttributeTok{title =} \StringTok{"Trend of Gender Parity Index in Liberia (Ages 15{-}24)"}\NormalTok{, }\AttributeTok{x =} \StringTok{"Year"}\NormalTok{, }\AttributeTok{y =} \StringTok{"Gender Parity Index(\%)"}\NormalTok{)}
\end{Highlighting}
\end{Shaded}

\includegraphics{P1-final_submission-template--1-_files/figure-latex/unnamed-chunk-7-2.pdf}

The red line graph illustrates an overall positive trend in the literacy
rates of young women in Liberia, despite some fluctuations. Initially,
the literacy rates show modest growth, followed by a significant rise
and a period of decline, before a strong recovery to the highest rates
observed towards the end of the period. This pattern reflects the
ongoing efforts and varying success in enhancing educational access and
quality for young women.

The blue line graph tracks the gender parity index, serving as a measure
of equality in educational access and achievement between genders. The
trend generally moves towards greater parity, indicating strides towards
equal educational opportunities for both genders. However, the journey
shows gradual progress with some setbacks, highlighting the intricate
challenges involved in consistently advancing gender parity in education
in Liberia.

\section{Infographic}\label{infographic}

\begin{quote}
I input the data for the year and values for both female literacy rates
and gender parity index into a Google Sheet and downloaded it as a CSV
file. I found a line graph template in Canva and uploaded the CSV file,
and the numbers were directly integrated into the line graph. I then
added all of the necessary text distributions such as the title, the
legends, and a brief explanation of the findings. I also included
visuals that represent a girl walking to school, which I colored pink
for the female literacy graph, and a boy and a girl standing equally to
represent gender parity in the gender parity graph.
\end{quote}

Blue Modern Line Chart Graph by Etta Brooks

\subsubsection{(c)}\label{c}

Provide a brief description of your data visualization process.

\begin{quote}
Data Preparation:
\end{quote}

Sourced and loaded data from the Humanitarian Data Exchange into R,
focusing on gender equality metrics specific to Liberia. Data Processing
in R:

Conducted data cleaning, transformation, and statistical analysis using
R packages like tidyverse and mosaic. Visualization:

Utilized ggplot in R for initial data visualization and then exported
key data for enhanced visual representation using Canva. Infographic
Enhancement:

Integrated the data into a Canva line graph template, enriching the
infographic with meaningful visuals and textual annotations to ensure
clarity and impact. Narrative and Presentation:

Carefully embedded the final infographic in the R Markdown document,
ensuring a cohesive narrative and visual flow.

\begin{center}\rule{0.5\linewidth}{0.5pt}\end{center}

\section{References}\label{references}

The creation of the infographic involved the use of various sources for
data, graphics, and analysis techniques. Below are the references used
in this project:

\begin{enumerate}
\def\labelenumi{\arabic{enumi}.}
\item
  \textbf{Data Source}: Humanitarian Data Exchange (HDX). Specific
  details about the gender equality metrics in Liberia. Available at
  \href{https://data.humdata.org/faq}{data.humdata.org}.
\item
  \textbf{R Packages}:

  \begin{itemize}
  \tightlist
  \item
    \texttt{tidyverse} for data manipulation and visualization.
  \item
    \texttt{mosaic} for statistical analysis.
  \end{itemize}
\item
  \textbf{Visualization Tool}:

  \begin{itemize}
  \tightlist
  \item
    Canva for enhancing data visualizations and creating infographics.
    Canva.
  \end{itemize}
\item
  \textbf{AI Assistance}:

  \begin{itemize}
  \tightlist
  \item
    OpenAI's ChatGPT was consulted for guidance on visualization
    techniques, as well as for proofreading and grammar suggestions.
  \end{itemize}
\end{enumerate}

\begin{center}\rule{0.5\linewidth}{0.5pt}\end{center}

\end{document}
